\documentclass[12pt]{article}
\usepackage{times}
\usepackage{ulem}
\usepackage[margin=1in]{geometry}
\usepackage{amsmath}


\begin{document}
\normalem

\setlength{\parindent}{0cm}
\title{\textbf{Algorithms for Identifying Contaminated Samples Genotyped with Single Nucleotide Polymorphisms}}
\author{\Large Elena Venable\\ 
	Healthy Oceans\\
	Eric C. Anderson\\
	Southwest Fisheries Science Center, Santa Cruz, CA\\
	National Oceanic and Atmospheric Adminstration (NOAA)
  \date{}}
\maketitle

%Abstract
\section*{Abstract} 
NOAA Fisheries laboratories have pioneered the use of single nucleotide polymorphism (SNP) genotypes to identify stocks of commercially
important fish species, such as Pacific salmon, and have demonstrated that SNPs outperform previously-used microsatellite genotypes on the
basis of cost, resolution, and ease of standardization. However, some geneticists have registered concerns that the biallelic nature of SNPs makes them unsuitable for identifying and removing contaminated samples. Here, we
present an accurate statistical method for identifying contaminated samples from SNP genotype data and tailor it to two data scenarios
commonly encountered in fisheries management: 1) sampling of genotypes for a single population, and 2) sampling of genotypes from a mixture
of individuals from different populations (such as in a mixed fishery). We present two different Markov chain Monte Carlo (MCMC) algorithms
for determining the proportion and identity of contaminated samples in single-population and mixture scenarios. We tested the 
single-population model using simulated data generated {\em in-silico} from observed Chinook salmon
(\textit{Oncorhynchus tshawytscha}) allele frequencies, and we tested the mixture model with simulations using a mixture of individuals
sampled from a previously genotyped Chinook salmon baseline data set. Using 60 or more loci, the single-population algorithm identified over
98 percent of contaminated samples while falsely identifying less than 0.28 percent of the non-contaminated samples.  Similarly, the mixture
model algorithm successfuly identified nearly all of the contaminated samples while producing few false positives. These results indicate
that our statistical method provides NOAA Fisheries scientists a valuable new tool to advance the role of cost-effective SNP genotypes in
fisheries management.

\end{document}
