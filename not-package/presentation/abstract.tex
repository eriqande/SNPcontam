\documentclass[12pt]{article}
\usepackage{times}
\usepackage{ulem}
\usepackage[margin=1in]{geometry}
\usepackage{amsmath}


\begin{document}
\normalem

\setlength{\parindent}{0cm}
\title{\textbf{Algorithms for Identifying Contaminated Samples Genotyped with Single Nucleotide Polymorphisms}}
\author{\Large Elena Venable\\ 
	Healthy Oceans\\
	Eric C. Anderson\\
	Southwest Fisheries Science Center, Santa Cruz, CA\\
	National Oceanic and Atmospheric Adminstration (NOAA)}
\maketitle

%Abstract
\section*{Abstract} 
NOAA Fisheries laboratories have pioneered the use of single nucleotide polymorphism (SNP) genotypes to identify stocks of commercially important fish species, such as Pacific salmon, and have demonstrated that SNPs outperform previously-used microsatellite genotypes on the basis of cost, resolution, and ease of standardization. However widespread adoption of SNP genotyping for fisheries management has been hindered by concerns that the biallelic nature of SNPs makes it difficult to identify contaminated samples from the genotype data. Here, we present an accurate statistical method for identifying contaminated samples from SNP genotype data and tailor it to two data scenarios commonly encountered in fisheries management: 1) sampling of genotypes for a single population, and 2) sampling of genotypes from a mixture of individuals from different populations (such as in a mixed fishery).   We present two different Markov chain Monte Carlo (MCMC) algorithms for determining the proportion of contaminated samples and identifying contaminated samples using genotype data, while also determining the allele frequencies from a single population and the mixture proportions from a mixture of populations.  We tested the single population and mixture models using simulated data from observed allele frequencies and baseline data respectively.  The simulation results from the single population model indicate that our algorithms identify over 98 percent of contaminated samples with over 60 loci, while falsely identifying less than 0.28 percent of samples. 

\end{document}
